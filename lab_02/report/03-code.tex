\chapter{Код программы}

\begin{lstlisting}[language=octave]
function main()
	f = fopen("X.txt", "r");
	buf = textscan(f, "%f", 'Delimiter', ' ');
	X = buf{1,1};
	X = X';
	fclose(f);
	
	gamma = 0.9;
	alpha = (1 - gamma) / 2;
	
	MU = MX(X);
	S2 = DX(X);
	fprintf("MU = %f, S2 = %f\n", MU, S2);
	
	MU_low = get_low_MU(X, alpha);
	MU_high = get_high_MU(X, alpha);
	fprintf("MU_low = %f, MU_high = %f\n", MU_low, MU_high);
	
	S2_low = get_low_S2(X, alpha);
	S2_high = get_high_S2(X, alpha);
	fprintf("S2_low = %f, S2_high = %f\n", S2_low, S2_high);
	
	gamma = 0.9;
	alpha = (1 - gamma) / 2;
	N = length(X);
	
	figure;
	hold on;
	plot([1, N], [MU, MU], 'r');
	plot((1 : N), get_MU_array(X, N), 'g');
	plot((1 : N), get_MU_low_array(X, N, alpha), 'b');
	plot((1 : N), get_MU_high_array(X, N, alpha), 'm');
	l = legend('\mu\^(x_N)', '\mu\^(x_n)', '_{--}\mu\^(x_n)', 
		'^{--}\mu\^(x_n)');
	set(l, 'fontsize', 14);
	grid on;
	
	figure;
	hold on;
	plot([2, N], [S2, S2], 'r');
	plot((1 : N), get_S2_array(X, N), 'g');
	plot((1 : N), get_S2_low_array(X, N, alpha), 'b');
	S2_high_array = get_S2_high_array(X, N, alpha);
	plot((4 : N), S2_high_array(4 : N), 'm');
	l = legend('S^2(x_N)', 'S^2(x_n)', '_{--}\sigma^2(x_n)', 
		'^{--}\sigma^2(x_n)');
	set(l, 'fontsize', 14);
	grid on;
end

function mx = MX(X)
	n = length(X);
	sum = sum(X);
	mx = sum / n;
endfunction

function dx = DX(X)
	n = length(X);
	MX = MX(X);
	if (n == 1)
		dx = 0;
	else
		dx = sum((X - MX).^2) / (n - 1);
	endif
endfunction

function mu_low = get_low_MU(X, alpha)
	n = length(X);
	MU = MX(X);
	S = sqrt(DX(X));
	mu_low = MU + S / sqrt(n) * tinv(alpha, n - 1);
endfunction

function mu_high = get_high_MU(X, alpha)
	n = length(X);
	MU = MX(X);
	S = sqrt(DX(X));
	mu_high = MU + S / sqrt(n) * tinv(1 - alpha, n - 1);
endfunction

function s2_low = get_low_S2(X, alpha)
	n = length(X);
	S2 = DX(X);
	s2_low = (n - 1) * S2 / chi2inv(1 - alpha, n -1);
endfunction

function s2_high = get_high_S2(X, alpha)
	n = length(X);
	S2 = DX(X);
	s2_high = (n - 1) * S2 / chi2inv(alpha, n -1);
endfunction

function MU_array = get_MU_array(X, N)
	MU_array = zeros(1, N);
	for i = 1 : N
		MU_array(i) = MX(X(1 : i));
	endfor
endfunction

function MU_low_array = get_MU_low_array(X, N, alpha)
	MU_low_array = zeros(1, N);
	for i = 1 : N
		cur_X = X(1 : i);
		MU_low_array(i) = get_low_MU(cur_X, alpha); 
	endfor
endfunction

function MU_high_array = get_MU_high_array(X, N, alpha)
	MU_high_array = zeros(1, N);
	for i = 1 : N
		cur_X = X(1 : i);
		MU_high_array(i) = get_high_MU(cur_X, alpha); 
	endfor
endfunction

function S2_array = get_S2_array(X, N)
	S2_array = zeros(1, N);
	for i = 1 : N
		S2_array(i) = DX(X(1 : i));
	endfor
endfunction

function S2_low_array = get_S2_low_array(X, N, alpha)
	S2_low_array = zeros(1, N);
	for i = 1 : N
		cur_X = X(1 : i);
		S2_low_array(i) = get_low_S2(cur_X, alpha);
	endfor
endfunction

function S2_high_array = get_S2_high_array(X, N, alpha)
	S2_high_array = zeros(1, N);
	for i = 1 : N
		cur_X = X(1 : i);
		S2_high_array(i) = get_high_S2(cur_X, alpha);
	endfor
endfunction
\end{lstlisting}
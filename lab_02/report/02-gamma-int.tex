\chapter{$\gamma$-доверительный интервал}

\section{Определение}

Пусть дана случайная величина $X$, закон распределения которой известен с точностью до неизвестного параметра $\theta$, и для параметра $\theta$ построен интервал $(\underline \theta (\vec X_n), \overline \theta (\vec X_n))$, где $\underline \theta (\vec x_n)$ и $\overline \theta (\vec X_n)$ -- пара статистик случайной выборки $\vec X_n$, такой что:
\begin{equation}
P\left\{\underline \theta (\vec X_n) < \theta < \overline \theta (\vec X_n)\right\} = \gamma.
\end{equation}
Тогда $(\underline \theta (\vec X_n), \overline \theta (\vec X_n))$ называется интервальной оценкой параметра $\theta$ с коэффициентом доверия $\gamma$ ($\gamma$-доверительный интервал).

Для удобства также можно ввести величину $\alpha = \displaystyle \frac{1 - \gamma}{2}$ -- вероятность отклонения результата в определенном направлении.

\section{Формулы для вычисления границ $\gamma$-доверительного интервала}

Пусть $\vec X_n$ -- случайная выборка объема $n$ из генеральной совокупности $X$, распределенной по нормальному закону с параметрами $\mu$ и $\sigma^2$.
\begin{enumerate}
	\item Для мат. ожидания:
	\begin{equation}
		\displaystyle \underline \mu(\vec X_n) = \overline X + \frac{S(\vec X_n)}{\sqrt{n}} t_{\alpha} (n - 1),
	\end{equation}
	\begin{equation}
		\displaystyle \overline \mu(\vec X_n) = \overline X + \frac{S(\vec X_n)}{\sqrt{n}} t_{1 - \alpha} (n - 1),
	\end{equation}
	где $\overline X$ -- точечная оценка мат. ожидания, $S(\vec X_n)$ -- точечная оценка дисперсии, $t_{1 - \alpha} (n - 1)$ -- квантиль уровня $\alpha$ распределения Стьюдента со степенями свободы $n - 1$;\newpage
	\item Для дисперсии:
	\begin{equation}
		\displaystyle \underline \sigma^2(\vec X_n) = \frac{(n - 1) \cdot S^2(\vec X_n)}{\chi_{1 - \alpha}^2 (n - 1)},
	\end{equation}
	\begin{equation}
		\displaystyle \overline \sigma^2(\vec X_n) = \frac{(n - 1) \cdot S^2(\vec X_n)}{\chi_{\alpha}^2 (n - 1)},
	\end{equation}
	где $S(\vec X_n)$ -- точечная оценка дисперсии, $\chi_{\alpha}^2 (n - 1)$ -- квантиль уровня $\alpha$ для распределения $\chi^2$ с $n - 1$ степенями свободы.
\end{enumerate}


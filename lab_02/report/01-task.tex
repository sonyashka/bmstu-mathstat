\chapter{Задание на лабораторную}

\textbf{\textit{Цель работы:}} построение доверительных интервалов для математического ожидания и дисперсии нормальной случайной величины.

\textbf{\textit{Содержание работы:}}
\begin{enumerate}
	\item Для выборки объема n из генеральной совокупности X реализовать в виде программы на ЭВМ:
	\begin{itemize}
		\item вычисление точечных оценок $\hat \mu (\vec x_n)$ и $S^2 (\vec x_n)$ математического ожидания MX и дисперсии DX соответственно;
		\item вычисление нижней и верхней границ $\underline \mu (\vec x_n)$, $\overline \mu (\vec x_n)$ для\newline $\gamma$-доверительного интервала для математического ожидания MX;
		\item вычисление нижней и верхней границ $\underline \sigma^2 (\vec x_n)$, $\overline \sigma^2 (\vec x_n)$ для\newline $\gamma$-доверительного интервала для дисперсии DX;
	\end{itemize}
	\item Вычислить $\hat \mu$ и $S^2$ для выборки из индивидуального варианта;
	\item Для заданного пользователем уровня доверия $\gamma$ и $N$ – объема выборки из индивидуального варианта:
	\begin{itemize}
		\item на координатной плоскости $Oyn$ построить прямую $y = \hat \mu (\vec x_N)$, также графики функций $y = \hat \mu (\vec x_n)$, $y = \underline \mu (\vec x_n)$, $y = \overline \mu (\vec x_n)$ как функций объема $n$ выборки, где $n$ изменяется от 1 до $N$;
		\item на другой координатной плоскости $Ozn$ построить прямую\newline $z = S^2(\vec x_N)$, также графики функций $z = S^2(\vec x_n)$, $z = \underline \sigma^2(\vec x_n)$, $z = \overline \sigma^2(\vec x_n)$ как функций объема $n$ выборки, где $n$ изменяется от 1 до $N$.
	\end{itemize}
\end{enumerate}

\textbf{\textit{В работе использовалась выборка по 13 варианту.}}
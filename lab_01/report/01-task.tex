\chapter{Задание на лабораторную}

\textbf{\textit{Цель работы:}} построение гистограммы и эмпирической функции распределения. 

\textbf{\textit{Содержание отчета:}}
\begin{enumerate}
	\item Для выборки объема n из генеральной совокупности X реализовать в виде программы на ЭВМ:
	\begin{itemize}
		\item вычисление максимального значения $M_{max}$ и минимального значения $M_{min}$;
		\item размаха $R$ выборки;
		\item вычисление оценок $\hat \mu$ и $S^2$ математического ожидания MX и дисперсии DX;
		\item группировку значений выборки в $m = [\log_2 n] + 2$ интервала;
		\item построение на одной координатной плоскости гистограммы и графика функции плотности распределения вероятностей нормальной случайной величины с математическим ожиданием $\hat \mu$ и дисперсией $S^2$;
		\item построение на другой координатной плоскости графика эмпирической функции распределения и функции распределения нормальной случайной величины с математическим ожиданием $\hat \mu$ и дисперсией $S^2$.
	\end{itemize}
	\item Провести вычисления и построить графики для выборки из индивидуального варианта.
\end{enumerate}

\textbf{\textit{В работе использовалась выборка по 13 варианту.}}
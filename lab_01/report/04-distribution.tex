\chapter{Эмпирическая функция распределения}

Пусть $\vec x = (x_1, ..., x_n)$ -- выборка из генеральной совокупности	$X$. Обозначим $n(x, \vec x)$ число элементов выборки $\vec x$, которые имеют значения меньше $x$.

Эмпирической функцией распределения называют функцию $F_n : \mathbb{R} \rightarrow \mathbb{R}$, определенную как:
\begin{equation}
F_n (x) = \frac{n(x, \vec x)}{n}.
\end{equation}

Замечание:
\begin{enumerate}
	\item Обладает всеми свойствами функции распределения;
	\item Кусочно-постоянна;
	\item Если все элементы вектора различны, то
	\begin{equation}
		F_n(x) = \begin{cases}
			0, & x \leq x_{(1)},\\
			\displaystyle \frac{i}{n}, & x_{(i)} < x \leq x_{(i + 1)},\\
			1, & x_{(n)} < x.
		\end{cases}
	\end{equation}
\end{enumerate}
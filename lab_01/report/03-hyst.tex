\chapter{Эмпирическая плотность и гистограмма}

Пусть $\vec x$ -- выборка из генеральной совокупности $X$. Если объем этой выборки $n$ велик, то ее значения $x_i$ группируют в интервальный статистический ряд. Для определения интервалов отрезок $J = [x_{(1)}, x_{(n)}]$ делят на $m$ равных частей, причем количество интервалов определяется как:
\begin{equation}
m = [\log_2 n] + 2.
\end{equation}

Сами интервалы определяются следующими выражениями:
\begin{gather}
J_i = [x_{(1)} + (i - 1) \cdot \Delta, x_{(1)} + i \cdot \delta), i = \overline {1, m - 1},\\
J_m = [x_{(1)} + (m - 1) \cdot \Delta, x_{(n)}],\\
\Delta = \frac{|J|}{m} = \frac{x_{(n)} - x_{(1)}}{m}.
\end{gather}

Интервальным статистическим рядом называется таблица:
\begin{figure}[H]
	\centering
	\begin{tabular}{|c|c|c|c|c|}
		\hline
		$J_1$ & ... & $J_i$ & ... & $J_m$ \\
		\hline
		$n_1$ & ... & $n_i$ & ... & $n_m$\\
		\hline
	\end{tabular}
\end{figure}
где $J_i$ -- $i$-ый полуинтервал статистического ряда, $n_i$ -- количество элементов выборки $\vec x$, попавших в $J_i$.

Эмпирическая плотность, отвечающая выборке $\vec x$ -- функция вида:
\begin{equation}
\hat f (x) = 
\begin{cases}
\displaystyle \frac{n_i}{n \Delta}, & x \in J_i, i = \overline{1, m},\\
0, & \text{иначе,}
\end{cases}
\end{equation}
где $J_i$ -- $i$-ый полуинтервал статистического ряда, $n_i$ -- количество элементов выборки $\vec x$, попавших в $J_i$, $n$ -- количество элементов выборки.

Гистограммой называется график эмпирической плотности.